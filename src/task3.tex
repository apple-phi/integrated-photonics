\documentclass[10pt, a4paper]{article}

% \usepackage[mathdisplays=normal]{savetrees}

% Core packages for better typography
\usepackage[utf8]{inputenc}
\usepackage[T1]{fontenc}
\usepackage{lmodern}
% \usepackage{newtxtext,newtxmath}
\usepackage{latexsym,amsfonts,amssymb,amsthm,amsmath}
\usepackage{mathtools}
\usepackage{textcomp}
\usepackage{microtype}
\usepackage{setspace}
\usepackage{siunitx}
\DeclareSIUnit{\um}{\ensuremath{\mu}\mathrm{m}}
% \onehalfspacing

% Math packages with better formatting

% Graphics and figures
\usepackage{tikz}
\usetikzlibrary{angles,quotes}
\usepackage{pgfplots}
\pgfplotsset{compat=newest}
\usepackage{graphicx}
\graphicspath{{../data/}}

% Better table formatting
\usepackage{booktabs}
\usepackage{array}
\usepackage{multirow}

% Force figures to stay in their sections
\usepackage[section]{placeins}
\makeatletter
\AtBeginDocument{%
  \expandafter\renewcommand\expandafter\subsection\expandafter{%
    \expandafter\@fb@secFB\subsection
  }%
}
\makeatother

% Code listing formatting
\usepackage{listings}
\usepackage{xcolor}
\definecolor{codegreen}{rgb}{0,0.6,0}
\definecolor{codegray}{rgb}{0.5,0.5,0.5}
\definecolor{codepurple}{rgb}{0.58,0,0.82}
\definecolor{backcolour}{rgb}{0.95,0.95,0.92}
\lstdefinestyle{mystyle}{
  backgroundcolor=\color{backcolour},   
  commentstyle=\color{codegreen},
  keywordstyle=\color{blue},
  numberstyle=\tiny\color{codegray},
  stringstyle=\color{codepurple},
  basicstyle=\ttfamily\footnotesize,
  breakatwhitespace=false,         
  breaklines=true,                 
  captionpos=b,                    
  keepspaces=true,                 
  showspaces=false,                
  showstringspaces=false,
  showtabs=false,                  
  tabsize=2
}
\lstset{style=mystyle}

% References and hyperlinks
\usepackage{pdfpages}
\usepackage[colorlinks=true, linkcolor=blue, citecolor=blue]{hyperref}
\usepackage{caption}
\usepackage{subcaption}
\usepackage{csquotes}
\usepackage[notes, backend=bibtex]{biblatex-chicago}
\addbibresource{refs.bib}



% Custom theorem environments
\newtheorem{theorem}{Theorem}[section]
\newtheorem{lemma}[theorem]{Lemma}
\newtheorem{proposition}[theorem]{Proposition}
\newtheorem{corollary}[theorem]{Corollary}
\newtheorem{definition}{Definition}[section]
\newtheorem{example}{Example}[section]

\title{\Large \bfseries SB4: Integrated Photonics\\[0.5em] \large Final Report}
\author{Lucas Ng\thanks{ln373@cam.ac.uk}}
\date{13th June 2025}

\begin{document}
\maketitle

\section{Introduction}

In the first interim report, we performed modal analysis on a symmetric planar waveguide
of several core and cladding materials, with further investigation into the effects of waveguide geometry and modal excitation wavelength on the modal properties,
including the effective index, group index, birefringence, and dispersion characteristics.

The second interim report recognised that waveguides do not exist in isolation,
and presented the coupled mode theory as a framework for understanding the power transfer in coupled waveguides,
with a derivation via considering the coupled mode propagation matrix,
\[
\mathbf{M} = \begin{pmatrix}
\beta & \kappa \\
\kappa & \beta
\end{pmatrix},
\]
to be the infinitessimal generator matrix that generates the modal evolution transition matrix as 
a one-parameter subgroup of the Lie group \(GL(2, \mathbb{C})\)
via the exponential map \(-jzM\mapsto e^{-jzM}\) (where \(z\) is the propagation distance along the waveguide)
that acts on an element of the Lie algebra \(\mathfrak{gl}(2, \mathbb{C})\).
This formalism demonstrated how the intrisic structure of \(M\) gives rise to the modal properties of the coupled waveguide:
the symmetry of the matrix \(M\) gives rise to reciprocal coupling,
while real-valued \(\beta\) and \(\kappa\) encodes the conservation of power in the coupled waveguide,
since in this instance, \(-jzM\) is Hermitian, and thus the image of the exponential map is unitary.

Likewise, this current, final report, ascends to the next level of abstraction,
\textit{\'a la mise en abyme},
considering the whole structure of a S-bend directional coupler in its entirety.
At this level, the previous theoretical frameworks serve purely as approximation,
and we must account for bending losses, asymmetric couplers, and other practical and interesting considerations.
This being analytically intractable, Lumerical's FDTD software\autocite{lumerical_fdtd} was utilised to simulate the directional coupler.

\section{Simulation setup}

The geometry of our directional coupler is the traditional doubly-symmetric S-bend design,
with a parallel coupling length connected in each waveguide to its input and output ports by S-bends.
Each port is extended by a straight waveguide which extends through to \SI{1}{\um} outside the FDTD simulation region,
which is otherwise internally padded on all sides by \SI{2.5}{\um} of perfectly matched layer (PML) boundary conditions.
The input port of the upper waveguide is excited by a modal source at the wavelength of interest.

This form of the directional coupler is, by way of metaphor, structured as an aubade.
It is parameterised by the widths of each waveguide (which may be varied independently),
the separation gap between the two waveguides, the length of the coupling region,
and the wavelength of the modal source.
The waveguide material is silicon with a cladding of silicon dioxide,
with wavelength-dependent refractive indices taken from the \texttt{Si (Silicon) - Palik}
and \texttt{SiO2 (Glass) - Palik} entries of the Lumerical material database respectively.
The waveguide depth into the plane is \SI{220}{\nm} for both waveguides,
the ports on each end are separated by \SI{10}{\um} in the horizontal direction,
and the poles of the S-bends are at \([(0,0), (5, 0), (\pm 5, \mp 5), (\pm 5, \mp 5)]\)\unit{\um}.

% TODO: Add a figure of the S-bend directional coupler geometry.

\section{Results}
\subsection{Equal power-splitting}

\subsection{Total power transfer}

\paragraph{At a wavelength of \(\lambda=\SI{1550}{\nm}\)}

\paragraph{At a wavelength of \(\lambda=\SI{1310}{\nm}\)}

\subsection{Unequal waveguide widths}

\section{Other contributions}


\section{Conclusion}


\printbibliography
\end{document}